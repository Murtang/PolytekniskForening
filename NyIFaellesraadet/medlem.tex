\section*{Om at være Fællesrådsmedlem}
\addcontentsline{toc}{section}{Om at være Fællesrådsmedlem}
Som dit faglige råds repræsentant i fællesrådet er det din opgave at repræsentere dit bagland bedst muligt. Mange af punkterne til fællesrådsmøderne vil blive diskuteret i de faglige råd og burde give dig en god ballast til afgive din stemme til fællesrådsmødet.\\

Det er dig som repræsentant der til mødet beslutter hvad der skal stemmes, PF er et repræsentativt demokrati og det faglige råd kan således ikke bestemme hvad du skal stemme til møderne.\\
Det faglige råd kan dog såfremt de er utilfredse med en eller flere af deres fællesrådsrepræsentanter fratage personen sin plads i fællesrådet of vælge en ny. Se § ???\\

Under selve mødet skal det altid meldes til referenten hvis man forlader lokalet permanent eller midlertidigt, ligesom man ved tilbagekomst eller sen ankomst også meddeler dette til referenten.
Snak i krogene frabedes, det kan man klare i pauserne.\\

FRFU udsender et foreløbigt referat senest tre uger efter at der er afholdt fællesrådsmøde. Dette referat bedes læst igennem hurtigst muligt og eventuelle sproglige, grammatiske eller navne fejl m.v. sendes med mail til FRFU der så kan nå at indføre rettelserne inden mødet. Indholdsmæssige ændringer til referatet bringes op til næste møde under godkendelse af referat.\\

Når der stemmes i fællesrådet foregår dette ved at række sin stemmeseddel i vejret. Armen skal være strakt og albuen højere oppe end skulderen. Dette gøres for at der ikke skal bruges unødigt meget tid på omtællinger fordi nogen synes det er nemmere at hvile armen på bordet og så lige vippe håndleddet når der er afstemning.\\

Ønsker man punkter behandlet på mødet må man meget gerne sende disse til FRFU på forhånd, eller annoncere dem til FRFU inden mødestart så punktet godkendelse af dagsorden kan klares nemmest.\\
I diskussioner anbefales det at alle så vidt muligt deltager således at det ikke er de samme tre personer eller råd der snakker mens resten er stille. Dog skal det ikke ses som en opfordring til at man bare gentager hvad andre allerede har sagt så diskussionen kører i ring, eller ved personvalg nævner hvem man støtter, dette kommer alligevel til udtryk under en afstemning.
\\
\textit{Skrevet af FRFU 2012}